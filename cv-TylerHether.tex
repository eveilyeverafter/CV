% LaTeX Curriculum Vitae Template
%
% Copyright (C) 2004-2009 Jason Blevins <jrblevin@sdf.lonestar.org>
% http://jblevins.org/projects/cv-template/
%
% You may use use this document as a template to create your own CV
% and you may redistribute the source code freely. No attribution is
% required in any resulting documents. I do ask that you please leave
% this notice and the above URL in the source code if you choose to
% redistribute this file.

\documentclass[letterpaper]{article}

\usepackage{hyperref}
\usepackage{geometry}

% Comment the following lines to use the default Computer Modern font
% instead of the Palatino font provided by the mathpazo package.
% Remove the 'osf' bit if you don't like the old style figures.
\usepackage[T1]{fontenc}
\usepackage[sc,osf]{mathpazo}

% Set your name here
\def\name{Tyler Duncan Hether}

% Replace this with a link to your CV if you like, or set it empty
% (as in \def\footerlink{}) to remove the link in the footer:
\def\footerlink{https://sites.google.com/site/buddingyeastbiologist/}

% The following metadata will show up in the PDF properties
\hypersetup{
  colorlinks = true,
  urlcolor = black,
  pdfauthor = {\name},
  pdfkeywords = {economics, statistics, mathematics},
  pdftitle = {\name: Curriculum Vitae},
  pdfsubject = {Curriculum Vitae},
  pdfpagemode = UseNone
}

\geometry{
  body={6.5in, 8.5in},
  left=1.0in,
  top=1.25in
}

% Customize page headers
\pagestyle{myheadings}
\markright{\name}
\thispagestyle{empty}

% Custom section fonts
\usepackage{sectsty}
\sectionfont{\rmfamily\mdseries\Large}
\subsectionfont{\rmfamily\mdseries\itshape\large}

% Other possible font commands include:
% \ttfamily for teletype,
% \sffamily for sans serif,
% \bfseries for bold,
% \scshape for small caps,
% \normalsize, \large, \Large, \LARGE sizes.

% Don't indent paragraphs.
\setlength\parindent{0em}

% Make lists without bullets
\renewenvironment{itemize}{
  \begin{list}{}{
    \setlength{\leftmargin}{1.5em}
  }
}{
  \end{list}
}

\begin{document}

% Place name at left
{\huge \name}

% Alternatively, print name centered and bold:
%\centerline{\huge \bf \name}

\vspace{0.25in}

\begin{minipage}{0.45\linewidth}
  \href{https://www.uidaho.edu/}{University of Idaho} \\
  Bioinformatics \& Computational Biology \\
  Life Sciences South 441D \\
  PO Box 443051, Moscow, ID. 83844-3051
\end{minipage}
\begin{minipage}{0.45\linewidth}
  \begin{tabular}{ll}
    Phone: & (208) 885-8860 \\
    Fax: &  (208) 885-7905 \\
    Email: & \href{mailto:tyler.thether@uidaho.edu}{\tt thether@uidaho.edu} \\
    Source Code: & \href{https://github.com/tylerhether}{\tt https://github.com/tylerhether} \\
  \end{tabular}
\end{minipage}


%\section*{Personal}

%\begin{itemize}
%\item Born on September 29, 1895.
%\item United States Citizen.
%\end{itemize}


\section*{Education}

\begin{itemize}
  \item Ph.D. Bioinformatics \& Computational Biology, University of Idaho. Expected May, 2016
    \begin{itemize}
    \item \emph{Dissertation:} ``The role of genetic interactions in adaptation''.
    \item \emph{Committee:} P. Hohenlohe, P. Joyce, C. Parent, \& J. Foster.
    \end{itemize}
  \item M.S. Biology, University of Central Florida.
    Received July, 2010
    \begin{itemize}
    \item \emph{Thesis:} ``Using landscape genetics to assess population connectivity in a habitat generalist''.
    \end{itemize}
  \item B.S. Biology, University of Central Florida.
    Received December, 2006
\end{itemize}


\section*{Employment}

\begin{itemize}
\item University of Idaho 2010--2015.
\item University of Central Florida 2006--2010.
\end{itemize}

\section*{Research Support}
\begin{itemize}
\item Eugene Magelby Natural Sciences Scholarship, 2015--2016
\item NSF DDIG ``The role of genetic interactions in adaptation'', 2014--2016
\item Bioinformatics \& Computational Biology Fellowship ``Develop computational tools to analyze recombination rate variation from low-coverage sequence data'', 2014--2016
\item NSF BEACON Graduate Fellowship ``The Genetic Architecture of Multi-dimensional Adaptation \& Speciation'', 2011--2013
\item Sigma Xi Grants-in-Aid of Research ``Evaluating the role of landscape features on gene flow'', 2008-2009
\end{itemize}

\section*{Programming \& Scripting Languages}
\begin{itemize}
\item R, C++, perl, python, \LaTeX , mathematica, \& bash
\end{itemize}

\section*{Publications}

\begin{itemize}
\item Hand, B.K, T.D. Hether, R.P. Kovach, C.C. Muhlfeld, S.J. Amish, M.C. Boyer, S.M. O'Rourke, M.R. Miller, W.H. Lowe, P.A. Hohenlohe, \& G. Lukart. 2015. Genomics and introgression: Discover and mapping of thousands of species-diagnoistic SNPs using RAD sequencing.  Current Zoology 61 (1): 146-154

\item Hether, T.D. and P.A. Hohenlohe.2014. Genetic regulatory network motifs constrain adaptation through curvature in the landscape of mutational (co)variance. Evolution (68) 4: 950-964.

\item Rosenblum, E.B., B.A. Sarver, J.W. Brown, S. Des Roches, K. M. Hardwick, T.D. Hether, J.M. Eastman, M.W. Pennell, and L.J. Harmon. 2011. Goldilocks meets Santa Rosalia: an ephemeral speciation model explains patterns of diversification across time scales. Evolutionary Biology. 39, number 2, 255-261.

\item Hether, T.D. and E.A. Hoffman. Machine learning identifies specific habitats associated with genetic connectivity in \emph{Hyla squirella}. 2012. J. Evolutionary Biology 25, issue 6, 1039-1052 

\item Degner, J.F., D.M. Silva, T.D. Hether, J.M. Daza, E.A. Hoffman. 2010. Fat frogs, mobile genes: unexpected phylogeographic patterns for the ornate chorus frog (\emph{Pseudacris ornata}). Molecular Ecology 19, issue 12, 2501-2515.

\item Jenkins D.G,..., T.D. Hether, et al. 2010. Isolation by distance: 20th century relic or reference standard for 21st century landscape genetics?  Ecography 33, issue 2, 315-320.

\item Hether, T. D. and E. A. Hoffman. Characterization of five dinucleotide and six tetranucleotide polymorphic microsatellite loci for the squirrel treefrog (\emph{Hyla squirella}).  Appeared in D. Abdoullaye, I. Acevedo, A.A. Adebayo, et al. 2010. Permanent Genetic Resources added to Molecular Ecology Resources Database 1 August 2009-30 September 2009. Molecular Ecology Resources 10, 232-236.

\item Degner, J. F., T. D. Hether, and E. A. Hoffman. 2009. Eight novel tetranucleotide and five cross-species dinucleotide microsatellite loci for the ornate chorus frog (\emph{Pseudacris ornata}). Molecular Ecology Resources 9, 622-624.


\end{itemize}

%\subsection*{Proceedings}

%\begin{itemize}
%\item A generalized T-Test and measure of multivariate dispersion,
%  Proc. Second Berkeley Symposium of Mathematical Statistics and
%  Probability, 1951.
%\end{itemize}


\section*{Talks \& Presentations}

\begin{itemize}
\item University of Idaho's IBEST Science Update. Talk title: Experimental Evolution of Local Adaptation in Yeast, 2014.
\item University of Idaho's College of Science Research Exposition 2014. Poster title: ``Hidden Markov Models Aid in Identifying Recombination Rate and Gene Conversion Hotspots in Low-Coverage DNA Sequencing of \textit{Saccharomyces cerevisiae} Crosses''. Primary author, 2014.
\item Evolution. Poster title: ``Genetic network architecture, mutation rate, and correlational selection affect G-matrix stability''. Primary author, 2014.
\item Evo-Wibo. Poster title: ``Genetic network architecture, mutation rate, and correlational selection affect G-matrix stability''. Primary author, 2014.
\item BEACON seminar series. Talk title:``Recombination rate and gene conversion heterogeneity: implications for `genomic islands' of divergence''. Primary author, 2013.
\item Evolution. \textit{Hamilton Award Nominee}. Presentation Title: ``Genetic regulatory motifs constrain adaptation through curvature in the landscape of mutational variation'' Primary author, 2013.
\item BEST Science Exposition, University of Idaho. Poster Title: ``Spatial autocorrelation structure of genomic sequence divergence under neutrality using coalescent simulations'' Contributing author, 2012.
\item BEST Science Exposition, University of Idaho, Idaho. Poster Title: ``Genetic Regulatory Networks, G-matrices, \& Adaptive Divergence'' Primary author, 2012.
\item Evolution. Poster Title: ``Genetic Network Architecture and the G-matrix Under Divergent Selection'' Primary author, 2012.
\item Seventh Annual Southeastern Ecology and Evolution Conference (SEEC), Atlanta, Georgia. Presentation Title: ``Using landscape genetics to evaluate habitat permeability in an abundant frog'' Primary author, 2010.
\item Biogeography Conference, Baja, Mexico. Poster Title: ``Isolation by distance is dead: long live IBD'' Contributing author, 2009.
\item Sixth Annual Southeastern Ecology and Evolution Conference (SEEC), Gainesville, Florida. Poster Title: ``Is clinal variation in skin color of the ornate chorus frog (\textit{Pseudacris ornata}) driven by natural selection?'' Primary author, 2009.
\item Evolution.  Presentation Title: ``Testing for selection along a cline of color change in a polymorphic frog'' Contributing author, 2008.
\end{itemize}

\section*{Educational Outreach}
\begin{itemize}
\item ``NIMBioS Evolutionary Quantitative Genetics Workshop'', 2015 -- Helped teach quantitative genetics to graduate students and postdocs
\item ``Creatures of the night'', 2011 -- Helped teach chiropteran natural history elementary childern
\item ``White Sands Institute'', 2011 -- Helped teach lizard evolution to middle school children
\item ``Save the Frogs Day'', 2011, Helped teach amphibian decline awareness and habitat requirements to preschool children. 
\end{itemize}

\section*{Journal Referee}
\begin{itemize}
\item Biological Journal of the Linnean Society, Evolution, \& Molecular Ecology
\end{itemize}

\bigskip

% Footer
\begin{center}
  \begin{footnotesize}
    Last updated: \today \\
    \href{\footerlink}{\texttt{\footerlink}}
  \end{footnotesize}
\end{center}

\end{document}
