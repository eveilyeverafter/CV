%% start of file `template.tex'.
%% Copyright 2006-2013 Xavier Danaux (xdanaux@gmail.com).
%
% This work may be distributed and/or modified under the
% conditions of the LaTeX Project Public License version 1.3c,
% available at http://www.latex-project.org/lppl/.


\documentclass[11pt,letterpaper,sans]{moderncv}        % possible options include font size ('10pt', '11pt' and '12pt'), paper size ('a4paper', 'letterpaper', 'a5paper', 'legalpaper', 'executivepaper' and 'landscape') and font family ('sans' and 'roman')

%\usepackage{apacite}
%\setlength{\biblabelsep}{10cm}
%\newcounter{MyBibCount}
%\makeatletter
%\renewcommand{\@biblabel}[1]{\textbf{\stepcounter{MyBibCount}}\theMyBibCount.}
%\makeatother

% modern themes
\moderncvstyle{banking}                            % style options are 'casual' (default), 'classic', 'oldstyle' and 'banking'
\moderncvcolor{blue}                                % color options 'blue' (default), 'orange', 'green', 'red', 'purple', 'grey' and 'black'
%\renewcommand{\familydefault}{\sfdefault}         % to set the default font; use '\sfdefault' for the default sans serif font, '\rmdefault' for the default roman one, or any tex font name
%\nopagenumbers{}                                  % uncomment to suppress automatic page numbering for CVs longer than one page

% character encoding
\usepackage[utf8]{inputenc}                       % if you are not using xelatex ou lualatex, replace by the encoding you are using
%\usepackage{CJKutf8}                              % if you need to use CJK to typeset your resume in Chinese, Japanese or Korean

% adjust the page margins
\usepackage[scale=0.85]{geometry}
%\setlength{\hintscolumnwidth}{3cm}                % if you want to change the width of the column with the dates
%\setlength{\makecvtitlenamewidth}{10cm}           % for the 'classic' style, if you want to force the width allocated to your name and avoid line breaks. be careful though, the length is normally calculated to avoid any overlap with your personal info; use this at your own typographical risks...

\usepackage{import}
\usepackage{pifont}
% personal data
\name{Tyler}{Hether}
\title{Resume }                               % optional, remove / comment the line if not wanted
\address{5289 University of Oregon // Eugene Oregon, 97403}{}{}% optional, remove / comment the line if not wanted; the "postcode city" and and "country" arguments can be omitted or provided empty
\phone[mobile]{Skype: Tyler.Hether}                   % optional, remove / comment the line if not wanted
\phone[fixed]{208 301 3259}                    % optional, remove / comment the line if not wanted
%\phone[fax]{+3~(456)~789~012}                      % optional, remove / comment the line if not wanted
\email{tyler.hether@gmail.com}                               % optional, remove / comment the line if not wanted
\homepage{tylerhether.github.io/}                         % optional, remove / comment the line if not wanted
% \extrainfo{Skype: Tyler.Hether}                 % optional, remove / comment the line if not wanted
%\photo[64pt][0.4pt]{picture}                       % optional, remove / comment the line if not wanted; '64pt' is the height the picture must be resized to, 0.4pt is the thickness of the frame around it (put it to 0pt for no frame) and 'picture' is the name of the picture file
%\quote{Some quote}                                 % optional, remove / comment the line if not wanted

% to show numerical labels in the bibliography (default is to show no labels); only useful if you make citations in your resume
%\makeatletter
%\renewcommand*{\bibliographyitemlabel}{\@biblabel{\arabic{enumiv}}}
%\makeatother
%\renewcommand*{\bibliographyitemlabel}{[\arabic{enumiv}]}% CONSIDER REPLACING THE ABOVE BY THIS

% bibliography with mutiple entries
%\usepackage{multibib}
%\newcites{book,misc}{{Books},{Others}}
%----------------------------------------------------------------------------------
%            content
%----------------------------------------------------------------------------------
\begin{document}
%\begin{CJK*}{UTF8}{gbsn}                          % to typeset your resume in Chinese using CJK
%-----       resume       ---------------------------------------------------------
\makecvtitle

\small{I'm a well-rounded data scientist whose interests are at the interface of data science, mathematics, and biology. I think my greatest attribute is the ability to quickly pick up new skills.}
% I'm a self-motivated, question driven and detail oriented person who enjoys working together with a team to create solutions to a wide-variety of biological and computational challenges. 
\section{Education}

\begin{itemize}
\item{
\centering
  \begin{tabular}{p{14.4cm}r}
\textbf{Ph.D, Bioinformatics \& Computational Biology} & University of Idaho, 2016\\
``Genetic Networks, Adaptation, \& the Evolution of Genomic Islands of Divergence'' & \\
  \end{tabular}
}

\item{
\centering
  \begin{tabular}{p{13cm}r}
\textbf{M.S., Biology} & University of Central Florida, 2010\\
``Using landscape genetics to assess population connectivity in a habitat generalist'' & \\
  \end{tabular}
}

\item{
\centering
  \begin{tabular}{p{13cm}r}
\textbf{B.S., Biology} & University of Central Florida, 2006\\
  \end{tabular}
}

\end{itemize}

\section{Employment}

%\vspace{4pt}
\begin{itemize}

\item{
\centering
  \begin{tabular}{p{15cm}r}
\textbf{University of Oregon} & Eugene, Oregon\\
\textit{Postdoctoral Associate \& Bioinformatics Data Scientist} & June 2016 -- present \\
  \end{tabular}
}
\\
To better foster scientific discovery across otherwise disparate research projects within the lab, I created and administer an Apollo (Jbrowse) genome browser on a cloud-based linux web server. I also provide bioinformatic support for lab members that range from designing experiments and providing statistical consultations to generating and implementing programing solutions to employing complete end-to-end data analysis.
\vspace{4pt}

\item{
\centering
  \begin{tabular}{p{14.2cm}r}
\textbf{University of Idaho} & Moscow, Idaho\\
\textit{Research Assistant \& Graduate Fellow} & August 2010 -- May 2016 \\
  \end{tabular}
}
\\
My dissertation examined the genomic response to environmental stress by experimentally evolving populations of budding yeast. Meiotic recombination plays a key role this response and we needed a find-scale landscape of recombination ``hotspots'' to make sense of the data. As a part of  my fellowship, I came up with a cost-effective way to infer genotypes from recombinant lines using sparse data (low-coverage sequence data). I created a Hidden Markov model solution that allowed us to save thousands of dollars in sequencing costs. I generalized my method and built an R package for use in the greater scientific community. 
\vspace{4pt}

\item{
\centering
  \begin{tabular}{p{14.2cm}r}
\textbf{University of Central Florida} & Orlando, Florida \\
\textit{Research Assistant \& Graduate Student} & August 2007 -- July 2010 \\
  \end{tabular}
}
\\
I isolated and characterized microsatellite markers for a range of species and identified landscape correlates to genetic connectivity between frog sub-populations using machine learning algorithms.

\end{itemize}

\vspace{4pt}

\section{Expertise and Interests}
Data Science \ding{118} Genomics \ding{118} Recombination \ding{118} Scalability \ding{118} HMMs  \ding{118} Genotype to Phenotype \ding{118} Molecular Evolution \ding{118} Experimental Evolution \ding{118} RNA sequencing \ding{118} R \ding{118} Rcpp \ding{118} Genetic Architecture \ding{118} Data Visualization \ding{118} Quantitative Genetics
%\vspace{5pt}

\section{Select Software on Github}


\begin{itemize}

\item{\textbf{HMMancestry.} \textit{'R package using the Forward-Backward algorithm to infer genotypes, recombination hotspots, and gene conversion tracts from low-coverage next-generation sequence data'}

\vspace{4pt}

\small{I created this package to infer recombination breakpoints, gene conversion tracts, hotspots, and coldspots in high-throughput, next-generation sequence data, even when sequencing coverage is relatively low. This package leverages nearby genetic content to infer local ancestry using a `Hidden Markov Model'. This package can analyze both haploid and diploid individuals and has built-in simulating and maximum-likelihood estimating functions for added user flexibility.}}

%\newpage

\item{\textbf{Flip2BeRAD.} \textit{'Python and C++ utilities for flipping RADseq reads'}

\vspace{4pt}

\small{I built a utility for flipping the forward and reverse raw reads generated from paired-end sequencing when the sample barcode is found on the reverse (paired-end) read. For some RADseq protocols (e.g., BestRAD), the barcode plus cut site combination can occur on the reverse read. This is problematic when downstream programs (e.g., Stacks) require that these be on the forward read. I built two flavors of Flip2BeRAD: a fuller featured Python script and a quicker C++ variant.}}

\vspace{4pt}

\item{\textbf{NetworkEvolution.}\textit{'Evolving networks in a quantitative genetics framework'}

\vspace{4pt}

\small{I created NetworkEvolution, a C++ program used to simulate two quantitative traits for a user-defined number of populations evolving to identical fitness optima under a quantitative genetics framework. A key feature of NetworkEvolution is the ability to simulate two classes of mutations: those in the allelic (coding) alleles and those in the cis-regulatory regions of a two gene genetic network.}}

\end{itemize}

\section{Technical \& Personal Skills}

\vspace{4pt}

\begin{itemize}

\item \textbf{Programming Languages.} In descending order of expertise: R, bash/linux, \LaTeX, C++, Python, MySQL, Mathematica, and Perl. Actively learning: JavaScript, Scala, php.

%\vspace{4pt}

%\item \textbf{Industry Software Skills.} Most MS Office products including MS Word and MS Excel. Cloud-based word processing (Google Drive suite of tools).

\vspace{4pt}

\item \textbf{Other.} Experience with high performance computing in R (via Rcpp) and reproducibility of documents, results, and reports using Sweave.  Experience in molecular and microbiology laboratory bench work. Experience presenting and disseminating findings at scientific conferences as well as in smaller groups and one-on-one.

\end{itemize}
\vspace{4pt}
%\nocite{*}
%\bibliographystyle{apacite}
%\bibliography{myrefs}
\section{Publications}
\begin{itemize}
\item \textbf{Hether, T.D.} and Hohenlohe P.A. Stochastic adaptation, overdominance, and reproductive isolation emerge from directional selection on simple genetic networks. \textit{In review}
\vspace{4pt}
\item \textbf{Hether, T.D.}, Wiench C.W., and Hohenlohe P.A. Novel molecular and analytical tools for efficient estimation of rates of meiotic crossovers and non-crossover gene conversions. \textit{In review}
\vspace{4pt}
\item Hand, B.K, \textbf{T.D. Hether}, R.P. Kovach, C.C. Muhlfeld, S.J. Amish, M.C. Boyer, S.M. O'Rourke, M.R. Miller, W.H. Lowe, P.A. Hohenlohe, \& G. Lukart. 2015. Genomics and introgression: Discover and mapping of thousands of species-diagnoistic SNPs using RAD sequencing.  Current Zoology 61 (1): 146-154
\vspace{4pt}
\item \textbf{Hether, T.D.} and P.A. Hohenlohe.2014. Genetic regulatory network motifs constrain adaptation through curvature in the landscape of mutational (co)variance. Evolution (68) 4: 950-964.
\vspace{4pt}
\item Rosenblum, E.B., B.A. Sarver, J.W. Brown, S. Des Roches, K. M. Hardwick, \textbf{T.D. Hether}, J.M. Eastman, M.W. Pennell, and L.J. Harmon. 2011. Goldilocks meets Santa Rosalia: an ephemeral speciation model explains patterns of diversification across time scales. Evolutionary Biology. 39, number 2, 255-261.
\vspace{4pt}
\item \textbf{Hether, T.D.} and E.A. Hoffman. Machine learning identifies specific habitats associated with genetic connectivity in \emph{Hyla squirella}. 2012. J. Evolutionary Biology 25, issue 6, 1039-1052
\vspace{4pt}
\item Degner, J.F., D.M. Silva, \textbf{T.D. Hether}, J.M. Daza, E.A. Hoffman. 2010. Fat frogs, mobile genes: unexpected phylogeographic patterns for the ornate chorus frog (\emph{Pseudacris ornata}). Molecular Ecology 19, issue 12, 2501-2515.
\vspace{4pt}
\item Jenkins D.G, ..., \textbf{T.D. Hether}, et al. 2010. Isolation by distance: 20th century relic or reference standard for 21st century landscape genetics?  Ecography 33, issue 2, 315-320.
\vspace{4pt}
\item \textbf{Hether, T. D.} and E. A. Hoffman. Characterization of five dinucleotide and six tetranucleotide polymorphic microsatellite loci for the squirrel treefrog (\emph{Hyla squirella}).  Appeared in D. Abdoullaye, I. Acevedo, A.A. Adebayo, et al. 2010. Permanent Genetic Resources added to Molecular Ecology Resources Database 1 August 2009-30 September 2009. Molecular Ecology Resources 10, 232-236.
\vspace{4pt}
\item Degner, J. F., \textbf{T. D. Hether}, and E. A. Hoffman. 2009. Eight novel tetranucleotide and five cross-species dinucleotide microsatellite loci for the ornate chorus frog (\emph{Pseudacris ornata}). Molecular Ecology Resources 9, 622-624.

\end{itemize}

\section{Other Interests \& Extracurricular Activities}

%\vspace{4pt}

\begin{itemize}

%\item{\textbf{Data transparency, open source code, and reproducibility.} The expression 'standing on the shoulders of giants' is most applicable in science and coding when you can actually use previous data and analytical tools.}

%\vspace{4pt}

\item{\textbf{Big data.} What happens when data gets really large and needs to be summarized in real-time? Recently, I've been learning how to use Apache Kafka and Twitter's streaming API to analyze gigabytes of local or streaming social media data efficiently. While it's a hobby of mine, these skills could have numerous industrial and biomedical applications.}

\vspace{4pt}

\item{\textbf{Data visualization.} Distilling large amounts of data down to make meaningful inferences is as much an art as it is a science. To this end, I employ a variety of visualization and statistical packages (e.g., ggplot2, dplyr) and JavaScript-based applications (e.g., \href{http://genomearchitect.github.io/}{Apollo}, \href{http://merenlab.org/software/anvio/}{anvi'o}, \href{http://catchenlab.life.illinois.edu/stacks}{Stacks}) in my work flow.}

\vspace{4pt}

%\item{\textbf{Outdoors.} While technology is great, I also like to get outdoors. Road cycling is one of my favorite activities. My favorite ride is the Trail of the Coeur d'Alene -- a pristine 72 mile bike trail that spans Idaho's panhandle from Washington to Montana where moose outnumber people. Hiking is another favorite pastime, both because it immerses one with nature as well as lends itself to great landscape photography opportunities.}

%\vspace{4pt}

% \item{Thing 3.I am also an avid hiker, having completed the national 3 peaks challenge last summer. Other interest include guitar, which I am self-taught, and home brewing.}

\end{itemize}





%-----       letter       ---------------------------------------------------------

\end{document}


%% end of file `template.tex'.
